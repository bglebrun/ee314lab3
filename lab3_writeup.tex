%%%%%%%%%%%%%%%%%%%%%%%%%%%%%%%%%%%%%%%%%
% Memo
% LaTeX Template
% Version 1.0 (30/12/13)
%
% This template has been downloaded from:
% http://www.LaTeXTemplates.com
%
% Original author:
% Rob Oakes (http://www.oak-tree.us) with modifications by:
% Vel (vel@latextemplates.com)
%
% License:
% CC BY-NC-SA 3.0 (http://creativecommons.org/licenses/by-nc-sa/3.0/)
%
%%%%%%%%%%%%%%%%%%%%%%%%%%%%%%%%%%%%%%%%%

\documentclass[letterpaper,11pt]{texMemo} % Set the paper size (letterpaper, a4paper, etc) and font size (10pt, 11pt or 12pt)

\usepackage{parskip} % Adds spacing between paragraphs
\usepackage[colorlinks]{hyperref}
\usepackage{graphicx}
\usepackage{float}
\hypersetup{citecolor=DeepPink4}
\hypersetup{linkcolor=red}
\hypersetup{urlcolor=blue}
\usepackage{cleveref}
\setlength{\parindent}{15pt} % Indent paragraphs

%----------------------------------------------------------------------------------------
%	MEMO INFORMATION
%----------------------------------------------------------------------------------------

\memoto{Dr.Shankarachary Ragi} % Recipient(s)

\memofrom{Benjamin LeBrun} % Sender(s)

\memosubject{Lab Assignment 3: Cyber Exploration Laboratory} % Memo subject

\memodate{\today} % Date, set to \today for automatically printing todays date

\logo{\includegraphics[width=0.1\textwidth]{logo.png}} % Institution logo at the top right of the memo, comment out this line for no logo

%----------------------------------------------------------------------------------------

\begin{document}

\maketitle % Print the memo header information

%----------------------------------------------------------------------------------------
%	MEMO CONTENT
%----------------------------------------------------------------------------------------

\section*{Introduction}
In this lab we were asked to verify a number of systems under different input effects: waveform input, 
loop gain, and system type upon steady state errors. This was accomplished using the in class lecture material 
in combination with MATLAB's control systems toolbox.

\section*{Prelab}
\subsection*{Questions}
\begin{enumerate}
    \item What system types will yeld zero steady-state error for step inputs?
    \item What system types will yield zero steady-state error for ramp inputs?
    \item What system types will yield infinite steady-state error for ramp inputs?
    \item What system types will yield zero steady-state error for parabolic inputs?
    \item What system types will yield infinite steady-state error for parabolic inputs?
    \item For the negative feedback system of Figure P7.35, where $G(s)= \frac{K(s+6)}{(s+4)(s+7)(s+9)(s+12)}$ and $H(s)=1$, calculate the steady-state error in terms of K for the following inputs: $5u(t)$,$5tu(t)$, and $5t^{2}u(t)$.
    \item Repeat prelab 6 for $G(s)=\frac{K(s+6)(s+8)}{s(s+4)(s+7)(s+9)(s+12)}$and $H(s)=1$.
    \item Repeat prelab 6 for $G(s)=\frac{K(s+6)(s+8)}{s^{2}(s+4)(s+7)(s+9)(s+12)}$and $H(s)=1$.
\end{enumerate}

\section*{Lab}
\begin{enumerate}
    \item Using Simulink, set up the negative feedback system of Prelab 6. Plot on one graph the error signal of the system for an input of $5u(t)$, and $K=50,500,1000$, and $5000$, repeat for inputs of $5t u(t)$ and $5t^{2}u(t)$.
    \item Using Simulink, set up the negative feedback system of Prelab 7. Plot on one graph the error signal of the system for an input of $5u(t)$, and $K=50,500,1000$, and $5000$, repeat for inputs of $5t u(t)$ and $5t^{2}u(t)$.
    \item Using Simulink, set up the negative feedback system of Prelab 8. Plot on one graph the error signal of the system for an input of $5u(t)$, and $K=200,400,800$, and $1000$, repeat for inputs of $5t u(t)$ and $5t^{2}u(t)$.
\end{enumerate}
%
%\begin{figure}[H]
%\begin{center}
%\includegraphics[width = 0.5\textwidth]{spare_me.jpg}
%\end{center}
%\caption{Don't forget to include images and pictures of your setup.}
%\label{fig:f4}
%\end{figure}

\section*{Postlab}
\begin{enumerate}
    \item Use your plots from Lab 1 and compare the expected steady-state errors to those calculated in the Prelab. Explain the reasons for any discrepancies.
    \item Use your plots from Lab 2 and compare the expected steady-state errors to those calculated in the Prelab. Explain the reasons for any discrepancies.
    \item Use your plots from Lab 3 and compare the expected steady-state errors to those calculated in the Prelab. Explain the reasons for any discrepancies.
\end{enumerate}

\section*{Appendices}
You'll want to put your full code here, possibly datasheets. Each appendix should start on a fresh page. Check out the \textbf{lstlisting} package, it might make your life easier. 
\newpage

\section*{Appendix A: Nothing}

%----------------------------------------------------------------------------------------

\end{document}
\grid
